% Options for packages loaded elsewhere
\PassOptionsToPackage{unicode}{hyperref}
\PassOptionsToPackage{hyphens}{url}
%
\documentclass[
  ignorenonframetext,
]{beamer}
\usepackage{pgfpages}
\setbeamertemplate{caption}[numbered]
\setbeamertemplate{caption label separator}{: }
\setbeamercolor{caption name}{fg=normal text.fg}
\beamertemplatenavigationsymbolsempty
% Prevent slide breaks in the middle of a paragraph
\widowpenalties 1 10000
\raggedbottom
\setbeamertemplate{part page}{
  \centering
  \begin{beamercolorbox}[sep=16pt,center]{part title}
    \usebeamerfont{part title}\insertpart\par
  \end{beamercolorbox}
}
\setbeamertemplate{section page}{
  \centering
  \begin{beamercolorbox}[sep=12pt,center]{part title}
    \usebeamerfont{section title}\insertsection\par
  \end{beamercolorbox}
}
\setbeamertemplate{subsection page}{
  \centering
  \begin{beamercolorbox}[sep=8pt,center]{part title}
    \usebeamerfont{subsection title}\insertsubsection\par
  \end{beamercolorbox}
}
\AtBeginPart{
  \frame{\partpage}
}
\AtBeginSection{
  \ifbibliography
  \else
    \frame{\sectionpage}
  \fi
}
\AtBeginSubsection{
  \frame{\subsectionpage}
}

\usepackage{amsmath,amssymb}
\usepackage{iftex}
\ifPDFTeX
  \usepackage[T1]{fontenc}
  \usepackage[utf8]{inputenc}
  \usepackage{textcomp} % provide euro and other symbols
\else % if luatex or xetex
  \usepackage{unicode-math}
  \defaultfontfeatures{Scale=MatchLowercase}
  \defaultfontfeatures[\rmfamily]{Ligatures=TeX,Scale=1}
\fi
\usepackage{lmodern}
\ifPDFTeX\else  
    % xetex/luatex font selection
\fi
% Use upquote if available, for straight quotes in verbatim environments
\IfFileExists{upquote.sty}{\usepackage{upquote}}{}
\IfFileExists{microtype.sty}{% use microtype if available
  \usepackage[]{microtype}
  \UseMicrotypeSet[protrusion]{basicmath} % disable protrusion for tt fonts
}{}
\makeatletter
\@ifundefined{KOMAClassName}{% if non-KOMA class
  \IfFileExists{parskip.sty}{%
    \usepackage{parskip}
  }{% else
    \setlength{\parindent}{0pt}
    \setlength{\parskip}{6pt plus 2pt minus 1pt}}
}{% if KOMA class
  \KOMAoptions{parskip=half}}
\makeatother
\usepackage{xcolor}
\newif\ifbibliography
\setlength{\emergencystretch}{3em} % prevent overfull lines
\setcounter{secnumdepth}{-\maxdimen} % remove section numbering


\providecommand{\tightlist}{%
  \setlength{\itemsep}{0pt}\setlength{\parskip}{0pt}}\usepackage{longtable,booktabs,array}
\usepackage{calc} % for calculating minipage widths
\usepackage{caption}
% Make caption package work with longtable
\makeatletter
\def\fnum@table{\tablename~\thetable}
\makeatother
\usepackage{graphicx}
\makeatletter
\def\maxwidth{\ifdim\Gin@nat@width>\linewidth\linewidth\else\Gin@nat@width\fi}
\def\maxheight{\ifdim\Gin@nat@height>\textheight\textheight\else\Gin@nat@height\fi}
\makeatother
% Scale images if necessary, so that they will not overflow the page
% margins by default, and it is still possible to overwrite the defaults
% using explicit options in \includegraphics[width, height, ...]{}
\setkeys{Gin}{width=\maxwidth,height=\maxheight,keepaspectratio}
% Set default figure placement to htbp
\makeatletter
\def\fps@figure{htbp}
\makeatother

\makeatletter
\@ifpackageloaded{caption}{}{\usepackage{caption}}
\AtBeginDocument{%
\ifdefined\contentsname
  \renewcommand*\contentsname{Table of contents}
\else
  \newcommand\contentsname{Table of contents}
\fi
\ifdefined\listfigurename
  \renewcommand*\listfigurename{List of Figures}
\else
  \newcommand\listfigurename{List of Figures}
\fi
\ifdefined\listtablename
  \renewcommand*\listtablename{List of Tables}
\else
  \newcommand\listtablename{List of Tables}
\fi
\ifdefined\figurename
  \renewcommand*\figurename{Figure}
\else
  \newcommand\figurename{Figure}
\fi
\ifdefined\tablename
  \renewcommand*\tablename{Table}
\else
  \newcommand\tablename{Table}
\fi
}
\@ifpackageloaded{float}{}{\usepackage{float}}
\floatstyle{ruled}
\@ifundefined{c@chapter}{\newfloat{codelisting}{h}{lop}}{\newfloat{codelisting}{h}{lop}[chapter]}
\floatname{codelisting}{Listing}
\newcommand*\listoflistings{\listof{codelisting}{List of Listings}}
\makeatother
\makeatletter
\makeatother
\makeatletter
\@ifpackageloaded{caption}{}{\usepackage{caption}}
\@ifpackageloaded{subcaption}{}{\usepackage{subcaption}}
\makeatother
\ifLuaTeX
  \usepackage{selnolig}  % disable illegal ligatures
\fi
\usepackage{bookmark}

\IfFileExists{xurl.sty}{\usepackage{xurl}}{} % add URL line breaks if available
\urlstyle{same} % disable monospaced font for URLs
\hypersetup{
  pdftitle={Classification Trees},
  hidelinks,
  pdfcreator={LaTeX via pandoc}}

\title{Classification Trees}
\author{}
\date{}

\begin{document}
\frame{\titlepage}

\begin{frame}{Quarto}
\phantomsection\label{quarto}
Quarto enables you to weave together content and executable code into a
finished presentation. To learn more about Quarto presentations see
\url{https://quarto.org/docs/presentations/}.
\end{frame}

\begin{frame}{Bullets}
\phantomsection\label{bullets}
When you click the \textbf{Render} button a document will be generated
that includes:

\begin{itemize}
\tightlist
\item
  Content authored with markdown
\item
  Output from executable code
\end{itemize}
\end{frame}

\begin{frame}[fragile]{Code}
\phantomsection\label{code}
When you click the \textbf{Render} button a presentation will be
generated that includes both content and the output of embedded code.
You can embed code like this:

\begin{verbatim}
[1] 2
\end{verbatim}
\end{frame}

\begin{frame}{Reading Materials}
\phantomsection\label{reading-materials}
\begin{itemize}
\tightlist
\item
  Max Kuhn. Chapter 14. Section 14.1
\end{itemize}
\end{frame}

\begin{frame}{Decision Boundary in Classification}
\phantomsection\label{decision-boundary-in-classification}
\includegraphics{images/db.png}

Classification is a process of finding the \textbf{decision boundary}
that best separate two classes
\end{frame}

\begin{frame}{Decision Boundary in Classification}
\phantomsection\label{decision-boundary-in-classification-1}
\includegraphics{images/db1.png}

SVM = Support Vector Machine
\end{frame}

\begin{frame}{Decision Tree}
\phantomsection\label{decision-tree}
\begin{itemize}
\tightlist
\item
  Decision Tree for classification is \textbf{Classification Tree}
\item
  Decision Tree for Regression is \textbf{Regression Tree}
\end{itemize}
\end{frame}

\begin{frame}{Example of Classification Tree}
\phantomsection\label{example-of-classification-tree}
\includegraphics{images/tree2.jpg}

{[}Link{]}
(http://graphics8.nytimes.com/images/2008/04/16/us/0416-nat-subOBAMA.jpg)
\end{frame}

\begin{frame}{Classification Tree}
\phantomsection\label{classification-tree}
\begin{itemize}
\tightlist
\item
  In two dimension, classification Tree's decision boundary is a
  collection of horizontal and vertical line
\end{itemize}
\end{frame}

\begin{frame}{Data}
\phantomsection\label{data}
\includegraphics{fall24_classification_tree_remake_quarto_test_files/figure-beamer/unnamed-chunk-2-1.pdf}

\begin{itemize}
\tightlist
\item
  The tree starts by a vertical or horizontal line that \textbf{best}
  seperate the data
\item
  \textbf{Question}: Find a vertical line that best seperate
  \textbf{red} and \textbf{green}.
\end{itemize}
\end{frame}

\begin{frame}{One way to seperate the reds and greens}
\phantomsection\label{one-way-to-seperate-the-reds-and-greens}
\includegraphics{fall24_classification_tree_remake_quarto_test_files/figure-beamer/unnamed-chunk-3-1.pdf}
\end{frame}

\begin{frame}{One way to seperate the reds and greens}
\phantomsection\label{one-way-to-seperate-the-reds-and-greens-1}
\includegraphics{fall24_classification_tree_remake_quarto_test_files/figure-beamer/unnamed-chunk-4-1.pdf}
\end{frame}

\begin{frame}{One way to seperate the reds and greens}
\phantomsection\label{one-way-to-seperate-the-reds-and-greens-2}
\includegraphics{fall24_classification_tree_remake_quarto_test_files/figure-beamer/unnamed-chunk-5-1.pdf}
\end{frame}

\begin{frame}{Question}
\phantomsection\label{question}
\begin{itemize}
\tightlist
\item
  \textbf{Question}: Which is the best split?
\end{itemize}
\end{frame}

\begin{frame}{Partial Answer}
\phantomsection\label{partial-answer}
\begin{itemize}
\tightlist
\item
  It looks like Split 1 and 3 are better than Split 2 since it
  misclassifies less
\item
  Which is the better split between Split 1 and Split 3?
\item
  We need to find a way to measure \emph{how good a split is}
\end{itemize}
\end{frame}

\begin{frame}{Impurity Measure}
\phantomsection\label{impurity-measure}
\begin{itemize}
\tightlist
\item
  The impurity of a node (\textbf{a node = a subset of the data or the
  original data}) measure how uncertain the node is.\\
\item
  For example, node A with 50\% reds and 50\% greens would be more
  uncertained than node B with 90\% reds and 10\% greens. Thus, node A
  has greater impurity than node B.
\item
  More uncertained \(=\) Greater impurity
\end{itemize}
\end{frame}

\begin{frame}{Impurity Measure}
\phantomsection\label{impurity-measure-1}
\begin{itemize}
\tightlist
\item
  A split that \emph{gains} more impurity is the \textbf{better split}!
\end{itemize}
\end{frame}

\begin{frame}{Impurity Gain}
\phantomsection\label{impurity-gain}
\includegraphics[width=\textwidth,height=1.5in]{images/parent_child.png}

\[
IG = I_{parent} - \frac{N_{left}}{N}I_{left}-\frac{N_{right}}{N}I_{right}
\]

\begin{itemize}
\tightlist
\item
  IG is Impurity Gain of the split
\item
  \(N_{left}\) and \(N_{right}\) are the number of points in the left
  child node and right child node, respectively.
\item
  \(N_{left}+N_{right}=N\)
\end{itemize}
\end{frame}

\begin{frame}{Impurity Measure}
\phantomsection\label{impurity-measure-2}
\begin{itemize}
\tightlist
\item
  Impurity can be measured by: classification error, Gini Index, and
  Entropy.
\end{itemize}
\end{frame}

\begin{frame}{Impurity Measure}
\phantomsection\label{impurity-measure-3}
\begin{itemize}
\tightlist
\item
  Let \(p_0\) and \(p_1\) be the proportion of class 0 and class 1 in a
  node.
\end{itemize}

\begin{align*}
{\text{By Classification Error: }} I &= min\{p_0, p_1\} \\
{\text{By Gini Index: }} I&= 1 - p_0^2-p_1^2 \\
{\text{By Entropy: }} I &= -p_0 \log_2(p_0)-p_1\log_2(p_1) 
\end{align*}
\end{frame}

\begin{frame}{Calculation}
\phantomsection\label{calculation}
\begin{itemize}
\tightlist
\item
  Let's calculate the impurity gain of the three splits to decide which
  split is the best
\end{itemize}
\end{frame}

\begin{frame}{IG By Classification Error}
\phantomsection\label{ig-by-classification-error}
\includegraphics[width=\textwidth,height=1in]{images/im1.png}

\begin{itemize}
\tightlist
\item
  Let \textbf{green} and \textbf{red} be class 0 and class 1,
  respectively.
\end{itemize}

For Split 1: \(N = 5, N_{left} =1, N_{right} = 4\)

\begin{itemize}
\item
  Node \emph{parent,} A: \(p_0 = \frac{2}{5}, p_1 = \frac{3}{5}\). Thus,
  \(I_{A} = \text{min}(\frac{2}{5}, \frac{3}{5}) = \frac{2}{5}\)
\item
  Node \emph{child left,} L:
  \(p_0 = \frac{0}{1} = 0, p_1 = \frac{1}{1} = 1\). Thus,
  \(I_{L} = \text{min}(0, 1) = 0\)
\item
  Node \emph{child right,} R: \(p_0 = \frac{3}{4}, p_1 = \frac{1}{4}\).
  Thus, \(I_{R} = \text{min}(\frac{3}{4}, \frac{1}{4}) = \frac{1}{4}\)
\item
  Impurity Gain of Split 1:
\end{itemize}

\[IG = \frac{2}{5} - \frac{1}{5} \cdot 0-\frac{4}{5} \cdot \frac{1}{4} = 0.2\]
\end{frame}

\begin{frame}{IG By Classification Error}
\phantomsection\label{ig-by-classification-error-1}
\includegraphics[width=\textwidth,height=1in]{images/im3.png}

For Split 2: \(N = 5, N_{left} =2, N_{right} = 3\)

\begin{itemize}
\item
  Node \emph{parent,} A: \(p_0 = \frac{2}{5}, p_1 = \frac{3}{5}\). Thus,
  \(I_{A} = \text{min}(\frac{2}{5}, \frac{3}{5}) = \frac{2}{5}\)
\item
  Node \emph{child left,} L: \(p_0 = \frac{1}{2}, p_1 = \frac{1}{2}\).
  Thus, \(I_{L} = \frac{1}{2}\)
\item
  Node \emph{child right,} R: \(p_0 = \frac{2}{3}, p_1 = \frac{1}{3}\).
  Thus, \(I_{R} = \text{min}(\frac{2}{3}, \frac{1}{3}) = \frac{1}{3}\)
\item
  Impurity Gain of Split 2:
\end{itemize}

\[IG = \frac{2}{5} - \frac{2}{5} \cdot \frac{1}{2}-\frac{3}{5} \cdot \frac{1}{3} = 0\]
\end{frame}

\begin{frame}{IG By Classification Error}
\phantomsection\label{ig-by-classification-error-2}
For Split 3: \(N = 5, N_{left} =3, N_{right} = 2\)

\includegraphics[width=\textwidth,height=1in]{images/im2.png}

\begin{itemize}
\item
  Node \emph{parent,} A: \(p_0 = \frac{2}{5}, p_1 = \frac{3}{5}\). Thus,
  \(I_{A} = \text{min}(\frac{2}{5}, \frac{3}{5}) = \frac{2}{5}\)
\item
  Node \emph{child left,} L: \(p_0 = \frac{1}{3}, p_1 = \frac{2}{3}\).
  Thus, \(I_{A} = \text{min}(\frac{1}{3}, \frac{2}{3}) = \frac{1}{3}\)
\item
  Node \emph{child right,} R: \(p_0 = \frac{2}{2}, p_1 = \frac{0}{2}\).
  Thus, \(I_{R} = \text{min}(1,0) = 0\)
\item
  Impurity Gain of Split 3:
\end{itemize}

\[IG = \frac{2}{5} - \frac{3}{5} \cdot \frac{1}{3}-\frac{2}{5} \cdot 0 = 0.2\]
\end{frame}

\begin{frame}{Comparing IG By Classification Error}
\phantomsection\label{comparing-ig-by-classification-error}
\begin{longtable}[]{@{}ll@{}}
\toprule\noalign{}
Split & IG \\
\midrule\noalign{}
\endhead
1 & 0.2 \\
2 & 0 \\
3 & 0.2 \\
\bottomrule\noalign{}
\end{longtable}

\begin{itemize}
\tightlist
\item
  By classification error, Split 1 and Split 3 are tie as the best
  because they have the same impurity gain.
\end{itemize}
\end{frame}

\begin{frame}{lol}
\phantomsection\label{lol}
\begin{itemize}
\tightlist
\item
  lol
\end{itemize}
\end{frame}



\end{document}
